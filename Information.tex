\section{Information Filter Basics}

Information filters turn the Kalman filter's system state
representation on its head by storing an information matrix $I_m$
which is the inverse of the $P$ matrix, and storing a vector $I_v$
which is the product of $I_m$ and the estimated state $x$.
When $I_m$ is invertable,
an estimate ${\tilde x}$ of the system state
and an estimate ${\tilde P}$ of the covaraince of the error
can be generated:
\begin{subequations}
\label{inforead}
\begin{align}
  {\tilde x} & = I_m^{-1} I_v
  & \mbox{\it (estimated State)}
\label{inforeadx}
\\
 {\tilde P} & = I_m^{-1}
  & \mbox{\it (covariance of estimated Error)}
\label{inforeadp}
\end{align}
\end{subequations}
See \verb|info_read.m|
to generate ${\tilde x}$ and ${\tilde P}$
from the filter state.

\subsection{Refinement}
Given an Information filter state
representing an estimate of the state of the system,
and an observation vector containing additional information
about the system,
a refined estimate of the state of the system
can be generated using the following equations:
\begin{subequations}
\label{infodata}
\begin{align}
  {\hat I_v} & = {\tilde I_v} + H^t R^{-1} z
  & \mbox{\it (improved State estimate)}
\label{infodatad}
\\
  {\hat I_m} & = {\tilde I_m} + H^t R^{-1} H
  & \mbox{\it (improved Error estimate)}
\label{infodatal}
\end{align}
\end{subequations}
See \verb|info_data.m|
for a Matlab implementation.

\subsection{Prediction}
Given an Information filter state
representing an estimate of the state of the system,
a prediction of the next state of the system
can be generated using the following equations:
\begin{subequations}
\label{infotime}
\begin{align}
  M & = F^{-T} I_m F^{-1}
\\
  C & = M (M + Q^{-1})^{-1}
\\
  L & = I - C
\\
  I_v & = L F^{-T} I_v
  & \mbox{\it (predicted State estimate)}
\label{infotimed}
\\
  I_m & = L M L^t + C Q^{-1} C^t
  & \mbox{\it (predicted Error estimate)}
\label{infotimel}
\end{align}
\end{subequations}
See \verb|info_time.m|
for a Matlab implementation.
The symbols $M$, $C$ and $L$
are used here as convenient names
for local scratch storage,
and have no significance
outside this specific procedure.
