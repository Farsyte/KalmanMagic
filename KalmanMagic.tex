\documentclass{article}
\usepackage{amsmath}
\numberwithin{equation}{section}

%% \parindent needs to be zero
%% \parskip may need to be increased.

\title{Kalman Filters\\and related magic}
\author{Greg Limes\\greglimes@gmail.com}

\begin{document}

\maketitle

\begin{abstract}
\noindent
At work,
we decided to use a Kalman filter
as our tool of choice
to estimate the states of
two testbed vehicles
based on prior estimates,
the known behavior of the systems,
and incoming noisy sensor observations.
The initial implementation
using the AutoBayes package
made use of a ``Bierman Update''
to process the sensor data,
which led me to a deeper study
of the Kalman filter and related methods.
This document attempts to capture
in distilled form
the highlights of what I have learned.
\end{abstract}

\newpage
\tableofcontents
\newpage
\input Conventions
\newpage
\input System
\newpage
\input Kalman
\newpage
\input Information
\newpage
\input Potter
\newpage
\input UDUt
\newpage
\input SRIF
\newpage
\appendix
\input AgeeTurner
\newpage
\input GramSchmidt
\newpage
\input Householder

\end{document}
