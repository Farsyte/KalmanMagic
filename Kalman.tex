\section{Kalman Filter Basics}

Standard Kalman Filters maintain an estimate $x$ of the state of the
system, and a matrix $P$ containing the covariance of the error of that
estimate. $P=0$ would represent absolute certainty that $x$ was the exact
state of the system. When even one element of the state is completely
unknown, infinite (or IEEE \verb|NaN|) values appear in $P$.

\subsection{Refinement}
Given a Kalman filter state
representing an estimate of the state of the system,
and an observation vector containing additional information
about the system,
a refined estimate of the state of the system
can be generated using the following equations:
\begin{subequations}
\label{kalmandata}
\begin{align}
  K & = {\tilde P} H^t (H {\tilde P} H^t + R)^{-1}
  & \mbox{\it (Kalman Gain)}
\label{kalmandatak}
\\
  {\hat x} & = {\tilde x} + K \left( z - H {\tilde x} \right)
  & \mbox{\it (improved State Estimate)}
\label{kalmandatax}
\\
  {\hat P} & = {\tilde P} - K H {\tilde P}
  & \mbox{\it (improved Error Estimate)}
\label{kalmandatap}
\end{align}
\end{subequations}
See \verb|kalman_data.m|
for a Matlab implementation.

\subsection{Prediction}
Given a Kalman filter state
representing an estimate of the state of the system,
a prediction of the next state of the system
can be generated using the following equations:
\begin{subequations}
\label{kalmantime}
\begin{align}
  {\tilde x} & = F {\hat x}
  & \mbox{\it (predicted State Estimate)}
\label{kalmantimex}
\\
  {\tilde P} & = F {\hat P} F^t + Q
  & \mbox{\it (predicted Error Estimate)}
\label{kalmantimep}
\end{align}
\end{subequations}
See \verb|kalman_time.m|
for a Matlab implementation.
