\section{Linear Discrete System Model}

All filters considered in this document
operate on a linear discrete model
of the system (plus noise), as expressed
by these equations:
\begin{subequations}
\label{sys}
\begin{align}
  z & = H x + N(0,R) & \mbox{\it (observation)}
\label{sysz}
\\
  y & = F x + N(0,Q) & \mbox{\it (process step)}
\label{sysx}
\end{align}
\end{subequations}

Some of the data processing algorithms require
that the observations are normalized so that
the sensor noise on each channel is independent,
gaussian, zero mean and unit variance.
A system with a general $R$ matrix can be normalized
by factoring $R = S_r S_r^t$ then inverting $S_r$ and
generating the normalized $H$ matrix (\ref{sysnormh})
and $z$ vector (\ref{sysnormz}).
\begin{subequations}
\label{sysnorm}
\begin{align}
  S_r S_r^t & = R & \mbox{\it (factor R)}
\\
  z & = H x + S_r N(0,I) & \mbox{\it (normalize noise)}
\\
  {\bar H} & = S_r^{-1} H & \mbox{\it (normalize H)}
\label{sysnormh}
\\
  {\bar z} & = S_r^{-1} z & \mbox{\it (normalize z)}
\label{sysnormz}
\end{align}
\end{subequations}

Some algorithms want to factor the $Q$ matrix
so that they are dealing with a diagonal matrix $Q_d$
representing the variances of the minimum set of
gaussian noise sources,
and a transfer matrix $S_q$ indicating how the
noise vector is transformed as it modifies
the system state, such that $Q = S_q Q_d S_q^t$.
Frequently, $Q_d$ and $S_q$ are directly constructed
from the design of the system.
\begin{subequations}
\label{sysqdiag}
\begin{align}
   S_q Q_d S_q^t & = Q & \mbox{\it (factor Q)}
\\
   y & = F x + S_q N(0,Q_d) & \mbox{\it (noise now independent)}
\label{sysxnorm}
\end{align}
\end{subequations}

Inclusion of control effects
is left as an exercise
for the implementation.
