\section{Householder Triangularization}

Several of the algorithms discussed in this document
require the partial triangularization of a matrix,
using orthogonal transformations;
they need the transformed matrix as a result
but do not need the transformation itself.

The method of choice to achieve this
is to apply Householder transformations
to the vectors appearing in the columns of the matrix,
selecting the transformation direction to
clear out data below the diagonal
in successive columns.

A typical partial triangularization might be
represented as follows:
\begin{equation}
T
\begin{bmatrix}
    A_{11} & A_{12}
\\  A_{21} & A_{22}
\end{bmatrix}
=
\begin{bmatrix}
    U_{11} & B_{12}
\\  0      & B_{22}
\end{bmatrix}
\label{householdertrim}
\end{equation}
where the resulting $U_{11}$ matrix
is upper triangular.

\subsection{Householder Vector Transformation}

The core of the algorithm is the Householder transform,
which generates the ``mirror image'' of an input vector $v$
in the ``plane'' perpendicular to parameter vector $u$,
calculated by measuring the extent of $v$
in the direction of $u$
and reversing it:
\begin{equation}
  v_u = v - 2 \left(\frac{v \cdot u}{u \cdot u}\right) u
\label{householderv}
\end{equation}

The transformation has many interesting properties,
but the most important property for us here now is
that it is an orthogonal transformation, so applying
any number of Householder transforms to all of the
column vectors in a matrix is still an orthogonal
transformation of the matrix.

The second interesting property is that the transformation
leaves unchanged any component in the original vector $v$
for which the corresponding component in the $u$ vector
is zero.

\subsection{Matrix Triangularization}

The partial triangularization will work across the matrix
column by column,
starting with the left column,
and continuing until the appropriate number of columns
have been triangularized.

Consider the vector present in the first column
of the input matrix.
We want to form the Householder transform
that will map this vector onto a new vector,
where all the components that fall below the diagonal
are zero.
Since this is an orthogonal transformation,
the sum of the squares of the diagonal element
and all elements below it must remain constant;
this condition is met by only two results,
which differ only in the sign of the element
that falls onto the diagonal of the matrix.

This gives us, for our input vector $v$,
two choices for the $v_u$ vector, and
some freedom in selecting the $u$ vector.
The simplest approach is to use
$u = v - v_u$
and to select (for numerical analysis reasons)
the sign of the new diagonal element
that avoids subtracting two large but
potentially nearly equal values.

Applying $T_u$ to the first column gives us
the desired result, with our selected value
on the diagonal to maintain the magnitude
of the vector, and zeros below it; we
then apply $T_u$ to the other columns.

We now focus on the next column, and
apply the same procedure,
paying careful attention to the wording above
relative to the elements of the vector
residing on, or below, the diagonal.
As the elements of $u$ that are ``above''
the diagonal are zero,
the transform $T_u$ will not modify
any data in the matrix above the diagonal
element under consideration;
and the data to the left of the current column
is also not modified by the transform
as already filled with zeros.

See \verb|hh_tri_c.c|, \verb|hh_tri_f.f| and \verb|hh_tri.m|
for implementations
of the Householder Partial Triangularization.

See \verb|hh_tri_quad.m| and \verb|hh_tri_quad_f.f|
for Matlab and Fortran mechanizations of
Householder Triangularization
where the matrix to be triangularized
is the concatination of four matrices
arranged in two rows of two matrices each.

See \verb|hh_tri_hex.m| and \verb|hh_tri_hex_f.f|
for Matlab and Fortran mechanizations of
Householder Triangularization
where the matrix to be triangularized
is the concatination of six matrices
arranged in two rows of three matrices each.
